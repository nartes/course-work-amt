\documentclass[unicode]{beamer}

\mode<presentation>
{
	\usetheme{Warsaw}
%	\setbeamertemplate{headline}{}
%	\setbeamertemplate{footline}{}
}

\usepackage[utf8]{inputenc}
\usepackage[T2A]{fontenc}
\usepackage{amssymb, amsmath, hyperref}
\usepackage[main=russian,english]{babel}

\title{Распознавание звука}
\author{Сергей Синяк}
\date{20.05.2016}

\begin{document}

\begin{frame}
	\titlepage
\end{frame}

\section{Исходная формулировка задачи}
\begin{frame}
  Человек исполнил композицию на инструменте. Аудио запись этого
  живого выступления подается на вход. Известно, что мелодия
  монофоническая, содержит не более 100 нот. Также на вход
  предоставляется набор шаблонов нот испольненных на этом
  инструменте.

  Необходимо распознать исходную нотную последовательность.
\end{frame}

\section{Проделанная работа}

\begin{frame}
   В результате было сделано:
  \begin{enumerate}
    \item Переведено доказтельство сходимости алгоритма NMF 
    \item Разобрано его доказательство с точностью до приведенного
      текста в работе
    \item Реализовано на практике в Octave его применение
      для простейшего полифонического сигнала из двух нот A4 и E4
    \item Сравнена работа этого алгоритма со встроенным в Octave
      из пакета linear-algebra на случайной матрице и на спектрограмме
      синтезированной аудиозаписи
    \item Реализовано на практике применение преобразования Фурье
      для спектрального анализа аудио записи и определения фундаментальной
      частоты
  \end{enumerate}
\end{frame}

\section{Пример}
\begin{frame}
\begin{figure}
    \includegraphics*[width=0.7\linewidth]
      {build/simple_analysis_dtfs.png}
    \caption{Матрица $V$ полученная функцией getspectrum}
\end{figure}
\end{frame}
\begin{frame}
\begin{figure}
    \includegraphics*[width=0.7\linewidth]
      {build/simple_analysis_nmf.png}
    \caption{Результат применения NMF алогритма}
      \label{F:SANMF}
\end{figure}
\end{frame}
\begin{frame}
NMF обозначает nonnegative matrix factorization.
С помощью преобразования Фурье получаются столбцы матрицы $V$.
Они представляют собой магнитудную спектрограмму.
Таким образом матрица $V$ отображаем изменение спектра с течением
времени. Затем решается задача
\[
  V \approx WH, \quad W,H \geqslant 0, \quad ||V-WH||  \;
  \text{-- функция потерь}
\]
То есть оптимизируется функция потерь -- Евклидова норма разности
$V$ и $WH$, таким образом строиться приближенное разложение $V$
в виде произведения матриц $W$ и $H$.
\end{frame}

\section{Заключение}

\begin{frame}
  \center{Спасибо за внимание!}
\end{frame}

\end{document}
