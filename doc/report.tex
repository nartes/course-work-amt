\documentclass[oneside, final, 14pt]{extarticle}
% rewrite with extreport
\usepackage[utf8]{inputenc}
\usepackage[russianb]{babel}
\usepackage[paper=a4paper, left=3cm, top=2cm, bottom=2cm, right=1cm]{geometry}
\usepackage{indentfirst}
\usepackage{amsmath}
\usepackage{amssymb}
\usepackage{verbatim}
\usepackage{ntheorem}
\usepackage{graphicx}
\usepackage{multicol}
\usepackage{algorithmic}
\usepackage{moreverb}

\begin{document}

\begin{titlepage}

%\newgeometry{margin=1cm}

\centerline{\bf МИНИСТЕРСТВО ОБРАЗОВАНИЯ РЕСПУБЛИКИ БЕЛАРУСЬ}
\bigskip
\bigskip
\centerline{\bf БЕЛОРУССКИЙ ГОСУДАРСТВЕННЫЙ УНИВЕРСИТЕТ}
\bigskip
\bigskip
\centerline{\bf ФАКУЛЬТЕТ ПРИКЛАДНОЙ МАТЕМАТИКИ И ИНФОРМАТИКИ}
\bigskip
\bigskip
\centerline{\bf Кафедра дискретной математики и алгоритмики}
\vfill
\vfill
\vfill
\centerline{\bf СИНЯК СЕРГЕЙ АЛЕКСАНДРОВИЧ}
\bigskip
\bigskip
\centerline{\large \bf РАСПОЗНАВАНИЕ ЗВУКА}
\vfill
\begin{centering}
  {
  Курсовая работа \\
  студента 3 курса, 3 группы, специальность <<информатика>> \\}
\end{centering}
\vfill
\vfill
\hfill
\begin{minipage}{0.35\textwidth}
  {\bf Научный руководитель \\
  {\small{\it Буславский Александр \\ Андреевич}}}
\end{minipage}
\vfill
\vfill
\centerline{\large \bf Минск 2016}

\restoregeometry

\end{titlepage}

\setcounter{page}{2}

\tableofcontents
\cleardoublepage

\section{Введение}
  Автоматическая транскрипция музыки, т.е. анализ записи музыкального
  произведения и генерация его нотной записи - партитуры -
  представляет собой очень интерестную задачу.

  Сама исходная задача впервые упоминалась в конце 1970-х годов. И к нашему
  времени существует множество частных решений данной проблемы. Большинство
  программных продуктов позволяют производить распознавание в
  полуавтоматическом режиме в силу допускаемых неточностей при распознавании.
  Особенно сложным является случай полифонических мелодий.  Для
  монофонических же мелодий точность существующих методов гораздо выше.

  В случае исходной постановки задачи требуется распознать нотную
  последовательность, а затем её воспроизвести. Дополнительно было указано,
  что длина не более 100 нот, мелодия может быть исполнена инструментом, либо
  непосредственно голосом человека, также имеется набор шаблонов нот.

  На данный момент требуется решить две проблемы, это разбиение мелодии на
  звуковые фрагменты соотвествующие отдельным нотам, а также определить
  фундаментальную частоту или высоту ноты.

  Для базового анализа применяется преобразование Фурье, которое позволяет
  разложить звук на частотный спектр. Для решения задачи временных
  интервалов нот, а также для определения их частотных профилей
  применяется NMF алгоритм. Которой осуществляет декомпозицию
  матрицы, столбцы которой есть спектрограммы полученные из исходного
  сигнала путём его разбиения на небольшие участки временным окном.
  В результате декомпозиции матрица представляется как произведение
  двух других. Одна из которых отражает частотные профили нот, а вторая
  содержит информацию о временных интервалах, в которые эти ноты звучат.
\cleardoublepage

\section{Некоторые сведения из теории музыки}

  В музыке, термин нота имеет два основных значения:

  \begin{enumerate}
    \item Знак используемый в музыкальной нотации для представления
      относительной длительности и высоты звука
    \item В смысле исполненного звука
  \end{enumerate}

  Две ноты с фундаментальной частотой с отношением равным некоторой
  степени двойки (например 1/2, 2, или 4) звучат доволе похожим
  образом. По этой причине, все ноты с отношением такого рода могут
  быть отнесены к одному классу по высоте.

  В традиционной теории музыки, большинство стран в мире используют
  соглашения именования До--Ре--Ми--Фа--Соль--Ля--Си. Но существует
  и другая нотация, которая использует первые 7 букв латинского
  алфавита A, B, C, D, E, F, G, где А --- это нота ля, остальные
  буквы в циклическом порядке соотвествуют обычной нотации.

  Восьмая нота, или октава, имеет такое же имя как и первая, но её частота
  удваивается. Имя октава также используется для указания дистанции между
  нотой и другой нотой с удвоенной частотой. Для различия двух нот из одного
  класса по высоте но расположенных в различных октавах, буква имени
  комбинируется с номером октавы. Например, 440 Гц именуется A4.

  Расстояние между двумя соседними нотами До--Ре, Ре--Ми, Фа--Соль, Соль--Ля,
  Ля--Си составляет тон, а между нотами Ми--Фа и Си--До - полутон. Отношение
  в полутон для частот нот выражется множителем $\sqrt[12]{2}$. Обычно
  шесть тонов октавы разбивают на 12 полутонов. Повышение или понижение
  ноты на полутон обозначается специальнами знаками $\sharp$ --- диез,
  $\flat$ --- бемоль. Например $A\sharp$, $E\flat$.

\cleardoublepage

\section{DSP}

\subsection{Цифровой сигнал}
  Сигнал представляет собой некоторую физическую величину, которая
  изменяется с течением времени и таким образом порождает функцию
  от времени. Физическая величина лежит в основе аналагового сигнала
  в основном. Также в силу непрерывности по природе своей таких сигналов
  они обычно непрерывные. Но существует и дискретные сигналы, которые
  могут иметь естветвенное происхождение, но также могут быть
  сгенерированными в результаты их цифровой обработки, ибо
  для работы компьютера необходимо производить дискретизацию по времени
  и по велечине сигналов. Это позволяет работать с ними как с обычными
  массивами данных. По окончанию процесса обработки результат может
  быть снова воссоздан в аналоговой форме, по необходимости.

  Как уже было упомянуто, в сигналах выделяется два измерения - время и
  значения, причём оба могут быть как непрерывные так и дискретные.

  В самой распостраненной форме сэмплирование, известное как периодическое
  или равномерное сэмплирование, это последовательность образцов
  $x[n]$, которая получена из непрерывного по времени сигнала
  $x_c(t)$ путём извлечения значения в раностоящих точках времени.

  Периодическое сэмплирование определяется как отношение
  \[
    x[n] \triangleq x_c(t) \left. \right|_{t=nT} = x_c(nT), \quad
    - \infty < n < \infty
  \]
  где $T$, фиксированный временное интервал между образцами, известно
  как период cэмплирования. Обратная велечина к периоду, $F_s = 1 / T$,
  называется частотой сэмплирования (когда выражается в оборотах в секунду)
  или скоростью сэмплирования (когда выражается в образцах в секунду).

\subsection{Преобразование Фурье}
  Цель анализа сигнала с помощью преборазования Фурье
  в том, чтобы разложить его на сумму синусоид.

  Отметим, что любая синусоида вида $A sin (2 \pi F_0 t + \Theta$
  может быть представлена суммой двух комплексных экспонент с одинаковой
  частотой:
  \[
    \Omega_0 = 2 \pi F_0 \quad
    A cos(\Omega_0 t + \Theta) =
    \frac{A}{2} e^{j\Theta} e^{j\Omega_0 t} +
    \frac{A}{2} e^{-j\Theta} e^{-j\Omega_0 t}
  \]

  Если предположить, что $t$ измеряется в секундах, то единица измерения
  $F_0$ есть колебания в секунду или Герцы (Гц).

  Для дискретного сигнала разложение делается по систему
  комплексных экспонент с гармонически связанными частоты и
  фундаментальной частотой $2 \pi / N$.

  Рассмотрим эту систему
  \[
    s_k[n]=e^{i\tfrac{2\pi}{N}kn}, \; k=\in{0,N-1}
  \]

  Для $s_k[n]$ выполняются следующие свойства

  \begin{enumerate}
  \item периодичность по времени \[
    s_k[n + N] = s_k[n],
  \]
  \item периодичность по частоте
  \[
    s_{k+N}[n] = s_k[n],
  \]
  \item ортогональность
  \[
    \sum_{k=0}^{N-1} s_k[n]s_m^*[n] =
    \sum_{k=0}^{N-1} e^{i\tfrac{2\pi}{N}kn} e^{i\tfrac{2\pi}{N}mn} =
    \left\{ \begin{aligned}
        N, \; k = m ,\\
        0, \; k \not= m.
      \end{aligned}
    \right.
  \]
  \end{enumerate}

  Рассмотрим фрагмент сигнала $s[n]$ где $n \in \{0,\cdots,N-1\}$.
  Представим его как линейную комбинацию $s_k[n]$.

  \[
    x[n] = \sum_{k=0}^{N-1} c_k e^{i\tfrac{2\pi}{N}kn}.
  \]

  Используя свойство ортогональности найдем неизвестные коэффициенты~$c_k$
  \[
    c_k = \dfrac{1}{N} \sum_{k=0}^{N-1} x[n] e^{-i\tfrac{2\pi}{N}kn}.
  \]

  График $x[n]$ как функции от времени дает описание сигнала во временной
  области. График $c_k$ как функции от частоты $F = k F_0$ (спектр)
  описывает сигнал в частотной области.

  Так как коэфициенты $c_k$ есть в общем случае комплексно-значные,
  то мы можем выразить их в полярной форме
  \[
    c_k = |c_k| e^{j\angle c_k}
  \]

  График $|c_k|$ известен как магнитудный спектр $x(t)$,
  при этом график $\angle c_k$ есть фазовый спектр $x(t)$.
  Если $c_k$ действительео-значные мы можем использвать единственный
  график известный как амплитудный спектр.

  Если $x(t)$ действительная функция от времени, то получаем
  \[
    c_{-k} = c_k^* = |c_k| e^{-j\angle c_k} \text{, отсюда}
  \]

  \[
    |c_k| = |c_{-k}|, \quad \angle (c_{-k}) = -\angle c_k
  \]
  откуда следует, что магнитудный спектр имеет четную симметрию
  и фазовый спектр нечетную симметрую.

\section{NMF}

\subsection{Постановка задачи}

Формулировка задачи неотрицательной факторизации матрицы (далее просто NMF)
следующая:

Для данной матрцы $V$ требуется найти неотрицательные множители $W$ и $H$
такие что
\[
  V \approx WH, \quad W,H \geqslant 0
\]

Теперь рассмотрим итеративный метод аппроксимации.
Но прежде надо определить некоторую функцию потерь.
Это может быть любая мера расстояния между матрицами.
Давайте рассмотрим Евклидовово расстояние:
\[
  ||A - B|| = \sqrt{\sum_{ij} (A_{ij} - B_{ij})^2}
\]

Это очевидно, что функция потерь не отрицательна
и обращается в 0 тогда и только тогда, когда $A = B$.

\textbf{Задача} Для данной константной матрицы $V$
и начальных значений переменных матриц $W$ и $H$
требуется минимизировать функцию потерь $||V-WH||$
с ограничениями $W,H \geqslant 0$.

\subsection{Градиентный спуск}
Давайте рассмотрим функцию $F = \frac{1}{2}||V - WH||^2$.

Её явная форма есть:
\[
  F = \frac{1}{2} \sum_{ij}
  \left(
    V_{ij} - (WH)_{ij}
  \right)^2
\]

Давайте определим $h_j$ как $j$-ый столбец матрцы $H$
и $w_i$ как $i$-ую строку матрицы $W$.

Перед введением мультипликативных правил перехода
для градиентного спуска давайте построим обычные правила.
Как это хорошо известно градиентный спуск идёт в направлении
обратном градиенту. Поэтому давайте получим формулу градиента
для $F$ относительно $h_j$ и $w_i$ отдельно.

\[
  F(h_j) = \frac{1}{2} \sum_{i}
  \left(
    V_{ij} - \sum_{k}W_{ik}H_{kj}
  \right)^2
\]

\begin{align*}
  \frac{\partial F(h_j)}{\partial H_{sj}}
  &=
  - \sum_i
    \left(
      V_{ij} - \sum_k W_{ik}H_{kj}
    \right)
  W_{is} \\
  &=
  -
    \left(
      \sum_i W_{is}V_{ij} - \sum_i\sum_kW_{ik}W_{is}H_{kj}
    \right)
\end{align*}

\[
  \nabla F(H) = - ( W^TV - W^TWH )
\]

\[
  F(w_i) = \frac{1}{2}\sum_j
    \left( V_{ij} - \sum_k W_{ik}H_{kj} \right)^2
\]

\begin{align*}
  \frac{\partial F(w_i)}{\partial W_{ia}}
  &=
  - \sum_j
    \left(
      V_{ij} - \sum_k W_{ik}H_{kj}
    \right)
  H_{aj} \\
  &=
  -
    \left(
      \sum_j H_{aj}V_{ij} - \sum_j \sum_k W_{ik}H_{kj}H_{aj}
    \right)
\end{align*}

\[
  \nabla F(W) = - ( VH^T - WHH^T )
\]

Тогда обычные правила градиентного спуска будут:
\begin{align*}
  H_{\alpha\mu} \leftarrow H_{\alpha\mu} +
  \eta_{\alpha\mu} \left[
     (W^TV)_{\alpha\mu} - (W^TWH)_{\alpha\mu}
  \right] \\
  W_{i\alpha} \leftarrow W_{i\alpha} +
  \zeta_{i\alpha} \left[
    (VH^T)_{i\alpha} - (WHH^T)_{i\alpha}
  \right]
\end{align*}

Здесь $\eta_{\alpha\mu}$ и $\zeta_{i\alpha}$ должны быть
такие параметры, что гарантируется сходимость итерационного
процесса.
Обычно для градиентного спуская они принимаются достаточно малыми.

\subsection{Мультипликативные правила перехода}

\textbf{Теорема} \textit{ Функция потерь $F$ не возрастает
при переходе
\[
  H_{\alpha\mu} \leftarrow H_{\alpha\mu}
  \frac{(W^TV)_{\alpha\mu}}{(W^TWH)_{\alpha\mu}} \quad
  W_{i\alpha} \leftarrow W_{i\alpha}
  \frac{(VH^T)_{i\alpha}}{(WHH^T)_{i\alpha}}
\]
Функция потерь $F$ инвариантна при таких правилах перехода тогда
и только тогда, когда $W$ и $H$ есть стационарные точки $F$.
}

\textbf{Простое замечание}

Мультипликативные правила быстрее обычных правил градиентного
спуска. Но они имеют недостаток, который выражается в знаменателе,
а именно $WH$. Когда процесс приближается к $V$ которая имеет нулевые
значения, то они -- нули -- появляются в знаменателе, ибо
$WH \approx V$, и это приводит
к {\itделению на ноль}.

$\square$


\cleardoublepage
\section{Реализация прототипа алгоритма в Octave}
  Программа Octave представляет собой OpenSource версию проприетарного приложения
  Matlab. В данном случае это позволяет визуализировать работу алгоритма с помощью
  встроенных средств построения графики. Что касается скорости работы, то это не
  принципиально и в конечном приложении производительность может быть увеличена
  засчет использования компилируемых языков типа C, C++.

  В приложении А приведен пример кода программы на Octave.

\cleardoublepage
\section{Практическое применения для анализа цифровой аудио записи}

  Для пробной проверки были сделана запись простой мелодии ля-до-ми в исполнении мужчины
  в малой октаве. Отметим, что запись производилась через микрофон ноутбука с
  достаточно плохим качеством записи. Ещё более усугубляющим фактором является
  работа бортового куллера на частоте порядка 100Гц.

  Искомые частоты нот будут соответственно A3 - 220Гц, C3 - 130Гц, E3 - 166Гц.

  Первоначально построим спектрограмму всего трека.
  \begin{figure}[h]
    \begin{multicols}{2}
      \hfill
      \includegraphics[width=80mm]{res/track.pdf}
      \hfill
      \caption{3D спектрограмма из Octave а}
      \label{pic_3da}
      \hfill
      \includegraphics[width=80mm]{res/track3d.pdf}
      \hfill
      \caption{3D спектрограмма из Octave б}
      \label{pic_3db}
    \end{multicols}
  \end{figure}

  Теперь выделим предполагаемый временной интервал для ноты ля и построим
  для него усредненную спектрограмму (Рис \ref{pic_p1_3d} $-$ \ref{pic_p1_ms}).

  \begin{figure}[t]
    \begin{multicols}{2}
      \hfill
      \includegraphics[width=80mm]{res/track_p1_3d_s.pdf}
      \hfill
      \caption{временной отрезок [0, 0.2], 3D спектрограмма }
      \label{pic_p1_3d}
      \hfill
      \includegraphics[width=80mm]{res/track_p1_ms.pdf}
      \hfill
      \caption{временной отрезок [0, 0.2], усредненная спектрограмма }
      \label{pic_p1_ms}
    \end{multicols}
  \end{figure}

  Как видим максимальным по амлпитуде является некоторый сигнал с частотой менее
  100Гц, следующим по убыванию можно выделить сигнал с частотой 210-220Гц.

  Если быть точным, то в окрестности частоты 210 усредненная спектрограмма содержит
  следующие значения:

  \begin{tabular}[t]{|c|c|c|c|c|c|c|c|c|c|}
    \hline
    $20 log_{10}|c_k|$ & -52 & -53 & -45 & -41 & -43 & -53 & -52 & -58 & -54 \\
    \hline
    $k Fs/N$ & 183 & 194 & 205 & 215 & 226 & 237 & 248 & 258 & 269 \\
    \hline
  \end{tabular}

  Попробуем уточнить результат увеличив ширину окна.

  Для удвоенного размера окна имеем:

  \begin{tabular}[t]{|c|c|c|c|c|c|c|c|c|c|}
    \hline
    $20 log_{10}|c_k|$ & -52 & -48 & -44 & -45 & -51 & -53 & -57 \\
    \hline
    $k Fs/N$ & 205 & 210 & 215 & 221 & 226 & 231 & 237 \\
    \hline
  \end{tabular}

  В случае 4-х кратного увелечения размера окна получаем:

  \begin{tabular}[t]{|c|c|c|c|c|c|c|c|}
    \hline
    $20 log_{10}|c_k|$ & -54 & -52 & -49 & -48 & -49 & -52 & -53 \\
    \hline
    $k Fs/N$ & 210 & 213 & 215 & 218 & 221 & 223 & 226 \\
    \hline
  \end{tabular}


  Получаем, что локальный максимум достигается при частоте 218Гц. Учитывая,
  что частоты соседних нот G3\# $-$ 208Гц, A3\# $-$ 233Гц, то это нота A3.

\cleardoublepage
\part{Использование NMF для распознавания музыки}
\section{NMF алгоритм}
\subsection{NMF}

Рассматривается алгоритм для решение следующей задачи:

Non-negative matrix factorization(NMF) Given a non-negative matrix
V, find non-negative matrix factors W and H such that:
\begin{equation}
    V \approx W H
\end{equation}

NMF can be applied to the statistical analysis of multivariate data
in the following manner.
Given a set of multivariate n-dimnesional data vectors, the vectors
are placed in the columns of an n x m matrix V
where m is the number of examples in the data set.
This matrix is then approximately factorized into an n x r matrix
W and an r x m matrix H.
Usually $r$ is chosen to be smaller than $n$ or $m$,
so that $W$ and $H$ are smaller than the original matrix $V$.
This results in a compressed version of the original data matrix.

Most significant idea in Eq.(1) is the following. It can be rewritten
column by column as $v \approx W h$, where $v$ and $h$ are the
corresponding columns of $V$ and $H$.
In other words, each data vector $v$ is approximated by a linear
combination of the columns of $W$, weighted by the components of $H$.
Therefore $W$ can be regarded as containing a basis
that is optimized for the linear approximation of the data in V.
Since relatively few basis vectors are used to represent
many data vectors, good approximation
can only be achieved if the basis vectors discover structure that is
latent in the data.

\subsection{Cost functions}

To find an approximate factorization $V \approx WH$, we first need
to define cost functions that quantify the quality of the
approximation. Such a cost function can be constructed using some
measure of distance between two non-negative matrices $A$ and $B$.
One useful measure is simply the square of the Euclidean distance
between $A$ and $B$,
\begin{equation}
  ||A - B||^2 = \sum_{ij}(A_{ij}-B_{ij})^2
\end{equation}

Рассматривается следующая формулировка NMF как задачи оптимизации:

Problem 1 Minimize $||V - W H ||^2$ with respect to $W$ and $H$,
subject to the constraints $W, H \geqslant 0$

Хотя функция $||V - WH||^2$ выпукла по отношению к $W$ только или
$H$ только, она не выкла по отношению к обоим переменным вместе.
Therefore it is unrealistic to expect an algorithm
to solve Problem $1$ in the sence of finding the sense of finding
global minima. However, there are many techniques from numerical
optimzation that can be applied to find local minima.

\subsection{Multiplicative update rules}

Theorem 1 The Euclidean distance $||V - WH||$ is nonincreasing under
the update rules
\begin{equation}
  H_{\alpha\mu} \leftarrow H_{\alpha\mu}
  \frac{(W^TV)_{\alpha\mu}}{(W^TWH)_{\alpha\mu}} \quad
  W_{i\alpha} \leftarrow W_{i\alpha}
  \frac{(VH^T)_{i\alpha}}{(WHH^T)_{i\alpha}}
\end{equation}

\subsection{Multplicative versus additive update rules}
It is useful to contrast these multiplicative updates with those
arising from gradient descent. In particular, a simple additive
update for $H$ that reduces the squared distance can be written
as
\begin{equation}
  H_{\alpha\mu} \leftarrow H_{\alpha\mu} + \gamma [
    (W^TV)_{\alpha\mu} - (W^TWH)_{\alpha\mu}
  ]
\end{equation}

If $\gamma$ are all set equal to some small positive number, this is
equivalent to conventional gradient descent. As long as this number
is sufficiently small. the update should reduce $||V-WH||$.

Now if we diagonally rescale the variables and set
\begin{equation}
  \gamma = \frac{H_{\alpha\mu}}{(W^TWH)_{\alpha\mu}}
\end{equation}
then we obtain the update rule for H that is given in Theorem 1.
Note that this rescaling results in a multiplicative factor with
the positive component of the gradient in the denominator
and the absolute value of the negative component in the numberator
of the factor.

\subsection{Proofs of convergence}
To prove Theorem 1, we will make use of an auxiliary function
similar to that used in the Expectation-Maximation algorithm.

Definition 1 $G(h, h')$ is an auxiliary function for $F(h)$ if the
conditions
\begin{equation}
  G(h,h') \geqslant F(h), \quad G(h,h) = F(H)
\end{equation}
are satisfied.

The auxiliary function is a usefull concept because of the following
lemma, which is also graphically illustrated in Fig. 4.

Lemma 1 if $G$ is an auxiliary functions, then $F$ is nonincreasing
under the update
\begin{equation}
  h^{t+1} = \text{arg min} G(h, h^t)
\end{equation}

Proof: $F(h^{t+1}) \geqslant G(h^{t+1},h^t) \geqslant G(h^t, h^t)
= F(h^t)$.

Note that $F(h^{t+1}) = F(h^t)$ only if $h^t$ is a local minimum of
$G(h,h^t)$. If the derivatives of $F$ exist and are continuous in
a small neighborhoood of $h^t$, this also implies that
the derivatives of $F$ equal 0. Thus, by iterating the update in
Eq. (7) we obtain a sequence of estimates that converge to a local
minimum $h_{min} = \text{arg min} F(h)$ of the objective function:
\begin{equation}
  F(h_{min}) \leqslant \cdots F(h^{t+1}) \leqslant F(h_2)
  \leqslant F(h_1) \leqslant (F(h_0)
\end{equation}.

\cleardoublepage

{\large \bf Заключение \\}
\addcontentsline{toc}{section}{Заключение}
  Был рассмотрен прототип алгоритма определения фундаментальной частоты фрагмента
  цифрового сигнала с помощью преобразования Фурье, который может быть
  использован для построения приложения распознавания последовательности нот.

\cleardoublepage
\addcontentsline{toc}{section}{\bibname}
\begin{thebibliography}{0}

  \bibitem{big_dsp_theory} Manolakis D. Applied digital signal
    processing: theory and practice, 2011 --- 157с.

  %\bibitem{dsp_matlab_practise} G. Blanchet and M. Charbit. Digital Signal
  %  and Image Processing using MATLAB, Volume 2: Advances and Applications:
  %  The Deterministic Case [2 ed.] 2014

  \bibitem{GDPE} Gerhard D., Pitch Extraction and Fundemental Frequency:
    History and Current Techniques, Technical Report TR-CS 2003-06, November, 2003

  \bibitem{GDCMA} Gerhard D., Computer Music Analysis, Techincal Report CMPT
    TR 97-13, October 13, 1997

  \bibitem{WMT} https://en.wikipedia.org/wiki/Transcription\_(music)

  \bibitem{WMP} https://en.wikipedia.org/wiki/Pitch\_(music)

  \bibitem{DDL} Daniel D. Lee, H. Sebastian Seung,
  Algorithms for Non-negative Matrix Factorization, 2000

  \bibitem{PMTPS} Paris Smaragdis, Judith C. Brown,
    Non-Negative Matrix Factorization
    for Polyphonic Music Trascription, October 19-22, 2003

  \bibitem{QUUX} http://www.quuxlabs.com/blog/2010/09/matrix-factorization-a-simple-tutorial-and-implementation-in-python/

\end{thebibliography}

\cleardoublepage
\appendix
{\large \bf Приложение A} \\
\listinginput[1]{1}{src/test2.m}

{\large \bf Приложение Б} \\
\listinginput[2]{1}{src/nmf.m}

\end{document}
% vi: tabstop=2 sw=2 sts=2
