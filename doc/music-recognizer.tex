\section{Music Recognizer}

To have a better platform for testing purposes a web-based application
has been developed. Its workflow for now corresponds to the Diagram \ref{F:6-sd}.

\begin{figure}[b]
  \centering
  \begin{tikzpicture}[every node/.style=draw]
    \matrix [matrix of nodes, column sep=5mm, row sep=5mm]
    {
      \node[align=center](a) { Microphone\\ pyaudio}; & \\
      \node[align=center](b) { WAV file\\ sms-tools}; &
          \node[align=center](d) {Audio signal\\ processing\\ pitch-estimator.py};
          & \node[align=center](e)
          {Fundamental frequency\\ estimation\\ Online-STFT.py}; \\
      \node[align=center](c)
          {Remote microphone\\ (NodeJS, WebSockets,\\ WebAudioAPI)};
          & \node[align=center](f) {Annotation generation\\ build/twm.txt};
          & \node[align=center](g) {Audio signal recording\\ build/output.wav}; \\
    };

    \draw [line width=0.3mm, ->] (a) -- (d);
    \draw [line width=0.3mm, ->] (b) -- (d);
    \draw [line width=0.3mm, ->] (c) -- (d);
    \draw [line width=0.3mm, ->] (d) -- (e);
    \draw [line width=0.3mm, ->] (e) -- (f);
    \draw [line width=0.3mm, ->] (e) -- (g);

  \end{tikzpicture}
  \caption{Structural diagram}
    \label{F:6-sd}
\end{figure}

\subsection{Fundamental frequency estimation using a two-way mismatch procedure}

Fundamental frequency ($F_0$) estimation for quasiharmonic signals
is an important task in music signal processing.
A \textit{two-way mismatch} (TWM) procedure for estimating $F_0$
is a computer-based method that uses the quasiharmonic assumption
to guide a search for $F_0$ based on the short-time spectra of an input signal.
The estimated $F_0$ is chosen to minimize discrepancies between measured
partial frequencies and harmonic frequencies generated by trial
values of $F_0$. For each trial $F_0$, mismatches between harmonics
generated and the measured partial frequencies are averaged over a fixed subset
of the available partials. A weighting scheme is used to reduce the
susceptibility of the procedure to the presence of noise or absence
of certain partials in the spectral data.

The algorithm is based on sinusoidal partials detection with a help of STFT.
After that for each frame a set of $F_0$ is estimated and one of the highest
rank is choosen as a resulting $F_0$.

The procedure uses the following error functions:
\begin{align}
  Err_{p \to m} &= \sum_{n=1}^N E_\omega (\Delta f_n, f_n, a_n, A_{max}) \\
                &= \sum_{n=1}^N \Delta f_n \times (f_n) ^{-p} +
                       (\frac{a_n}{A_{max}}) \times [q \Delta f_n \times
                       (f_n)^{-p} - r],
\end{align}

where $\Delta f_n$ is difference between predicted and the closest measured
peaks, $f_n, a_n$ are frequency and magnitude of predicted peaks,
$A_{max}$ is maximum peak magnitude.

\begin{align}
  Err_{m \to p} &= \sum_{k=1}^K E_\omega (\Delta f_k, f_k, a_k, A_{max}) \\
                &= \sum_{k=1}^K \Delta f_k \times (f_k) ^{-p} +
                       (\frac{a_k}{A_{max}}) \times [q \Delta f_k \times
                       (f_k)^{-p} - r],
\end{align}

where $\Delta f_k$ is difference between measured and the closest predicted
peaks, $f_n, a_n$ are frequency and magnitude of measured peaks,
$A_{max}$ is maximum peak magnitude.

\textbf{Total error:} $Err_{total} = Err_{p \to m}/N + \rho Err_{m \to p}/K$.

Maher and Beauchamp propose $p=0.5, q=1.4, r=0.5, \rho=0.33$.
