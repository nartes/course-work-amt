\section{Music Recognizer}

Для лучшего тестирования было разработано веб-приложение, структурная схема
которого представлена на Диаграмме \ref{F:6-sd}.

\begin{figure}[b]
  \centering
  \begin{tikzpicture}[every node/.style=draw]
    \matrix [matrix of nodes, column sep=5mm, row sep=5mm]
    {
      \node[align=center](a) { Микрофон\\ pyaudio}; & \\
      \node[align=center](b) { WAV файл\\ sms-tools}; &
          \node[align=center](d) {Обработка\\ аудио сигнала\\ pitch-estimator.py};
          & \node[align=center](e)
          {Определение\\ фундаментальной частоты\\ Online-STFT.py}; \\
      \node[align=center](c)
          {Удалённый микрофон\\ (NodeJS, WebSockets,\\ WebAudioAPI)};
          & \node[align=center](f) {Генерация аннотации\\ в build/twm.txt};
          & \node[align=center](g) {Запись аудио сигнала\\ в build/output.wav}; \\
    };

    \draw [line width=0.3mm, ->] (a) -- (d);
    \draw [line width=0.3mm, ->] (b) -- (d);
    \draw [line width=0.3mm, ->] (c) -- (d);
    \draw [line width=0.3mm, ->] (d) -- (e);
    \draw [line width=0.3mm, ->] (e) -- (f);
    \draw [line width=0.3mm, ->] (e) -- (g);

  \end{tikzpicture}
  \caption{Структурная диаграмма}
    \label{F:6-sd}
\end{figure}

\subsection{Определение фундаментальной частоты используя \\
процедуру two-way mismatch}

Определение фундаментальной частоты в квазигармонических сигналах
является важной задачей в обработке музыкальных сигналов.
Процедура \textit{two-way mismatch} (TWM) определения $F_0$ -
это компьютерный метод, который опирается на квазигармоничесность,
что позволяет находить $F_0$ на основе кратковременного спектра входного
сигнала.
Найденная $F_0$ выбирается так, что минимизируется расхождение между
измеренными частотными (синусоидальными) компонентами и гармоническими
компонентами сгенерированными на основе текущего кандидата для $F_0$.
Для каждой пробы $F_0$, несовпадения между сгенерированными гармониками
и измеренными частотными компонентами усредняются на фиксированном
подмножестве доступных компонент. Схема взвешивания используется для
снижения чувствительности процедуры к наличию шума или отсутствию
определенных компонент в спетральных данных.

Алгоритм основан на определении синусоидальных компонент с помощью STFT.
После этого для каждого окна определяется множество $F_0$ и как результат
выбирается та, у которой минимальная ошибка.

Процедура использует следующие функции ошибки:
\begin{align}
  Err_{p \to m} &= \sum_{n=1}^N E_\omega (\Delta f_n, f_n, a_n, A_{max}) \\
                &= \sum_{n=1}^N \Delta f_n \times (f_n) ^{-p} +
                       (\frac{a_n}{A_{max}}) \times [q \Delta f_n \times
                       (f_n)^{-p} - r],
\end{align}

где $\Delta f_n$ есть разница между предсказанными и наиближайшеми измеренными
пиками, $f_n, a_n$ есть частота и магнитуда предсказанных пиков,
$A_{max}$ есть максимальная магнитуда среди пиков.

\begin{align}
  Err_{m \to p} &= \sum_{k=1}^K E_\omega (\Delta f_k, f_k, a_k, A_{max}) \\
                &= \sum_{k=1}^K \Delta f_k \times (f_k) ^{-p} +
                       (\frac{a_k}{A_{max}}) \times [q \Delta f_k \times
                       (f_k)^{-p} - r],
\end{align}

где $\Delta f_k$ есть разница между измеренными и наиближайшеми предсказанными
пиками, $f_k, a_k$ есть частота и магнитуда измеренных пиков,
$A_{max}$ есть максимальная магнитуда среди пиков.

\textbf{Суммарная ошибка:} $Err_{total} = Err_{p \to m}/N + \rho Err_{m \to p}/K$.

Maher и Beauchamp предлагают следующие значения для параметров
$p=0.5, q=1.4, r=0.5, \rho=0.33$.
